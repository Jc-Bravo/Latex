%!xelatex = 'xelatex --halt-on-error %O %S'

\documentclass{thuemp}
\begin{document}

% 标题,作者
\emptitle{The AI cannot replace humans forever}
\empauthor{Jacob}{2019011325}
\fancyhead[CO]{{\footnotesize Jacob: Academic English}}
\maketitle

%%%%%%%%!首页角注可能与正文重叠,请通过调整正文中第一页的\enlargethispage{-3.3cm}位置手动校准正文底部位置:
%%%%%%%%%%%%%%%%%%%%%%%%%%%%%%%%%%%%%%%%%%%%%%%%%%%%%%%%%%%%%%%%
%  正文由此开始
\wuhao 
%  分栏开始
\section*{introductory statement}

Artificial intelligence (AI) is intelligence demonstrated by
machines, as opposed to the natural intelligence displayed by
animals, including humans.\cite{enwiki:1084791620}
This work culminated in the invention of the programmable digital computer in the 1940s, 
the machine based on the abstract essence of mathematical reasoning. \cite{AI:found}. 
Formal design for Turing-complete "artificial neurons," but it is seen that it does not look like it is generating any value. 
In 1956, called an "AI winter"\cite{AI:winter}.
With slow progress, until 1985, when the market for AI had reached the first billion dollars. \cite{AI:spring}.
From now on. with the growth of the highly mathematical-statistical machine learning,
 AI has been famous in many places, such as in understanding human speech
 (Siri);self-driving cars(Tesla);web search engines
 (Google, Baidu, Bing) and many other places.
 With the AI development,
they learn fast and fast; there comes a question: Will artificial intelligence replace humans?
This essay will argue that while AI is fast developing, AI cannot replace humans forever. 
Many occupations have been replaced due to the rapid development of artificial intelligence. 
However, as a technology, artificial intelligence can only be used to improve the efficiency of doing things and cannot replace human beings.
Much evidence shows that Ai will Eliminate some backward industries, but many are irreplaceable by artificial intelligence.

\section*{Thesis}

\section*{Some of the following}

- context

- definition

- limited scope

\section*{(optional) an essay map}







%%%%%%%%%%%%%%%%%%%%%%%%%%%%%%%%%%%%%%%%%%%%%%%%%%
%  参考文献
%%%%%%%%%%%%%%%%%%%%%%%%%%%%%%%%%%%%%%%%%%%%%%%%%%%%%%%%%%%%%%%%
%  参考文献按GB/T 7714-2015《文后参考文献著录规则》的要求著录. 
%  参考文献在正文中的引用方法:\cite{bib文件条目的第一行}

\renewcommand\refname{\heiti\wuhao\centerline{References}\global\def\refname{References}}
\vskip 12pt

\let\OLDthebibliography\thebibliography
\renewcommand\thebibliography[1]{
  \OLDthebibliography{#1}
  \setlength{\parskip}{0pt}
  \setlength{\itemsep}{0pt plus 0.3ex}
}

{
\renewcommand{\baselinestretch}{0.9}
\liuhao
\bibliographystyle{gbt7714-numerical}
\bibliography{./TempExample}
}


\end{document}

grammarly.check()
