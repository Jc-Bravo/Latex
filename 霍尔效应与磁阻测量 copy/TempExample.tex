%!xelatex = 'xelatex --halt-on-error %O %S'

\documentclass{thuemp}
\begin{document}

% 标题,作者
\emptitle{关于斐波那契数列计算的方法效率比较}
\empauthor{江灿}{2019011325}

% 奇数页页眉 % 请在这里写出第一作者以及论文题目
\fancyhead[CO]{{\footnotesize 江灿: 关于斐波那契数列计算的方法效率比较}}


%%%%%%%%%%%%%%%%%%%%%%%%%%%%%%%%%%%%%%%%%%%%%%%%%%%%%%%%%%%%%%%%
% 关键词 摘要 首页脚注
%%%%%%%%关键词
\Keyword{斐波那契,计算,效率,对比}
\twocolumn[
\begin{@twocolumnfalse}
\maketitle

%%%%%%%%摘要
\begin{empAbstract}
斐波那契作为一个
\end{empAbstract}

%%%%%%%%首页角注,依次为实验时间、报告时间、学号、email
\end{@twocolumnfalse}
1
%%%%%%%%!首页角注可能与正文重叠,请通过调整正文中第一页的\enlargethispage{-3.3cm}位置手动校准正文底部位置:
%%%%%%%%%%%%%%%%%%%%%%%%%%%%%%%%%%%%%%%%%%%%%%%%%%%%%%%%%%%%%%%%
%  正文由此开始
\wuhao 
%  分栏开始

\section{实~~验~~目~~的}
\subsection{了解霍尔效应的产生原理以及副效应的产生原理}
\subsection{掌握霍尔系数的测量方法,学习消除霍尔效应的实验方法}
\subsection{研究半导体材料的电阻值随磁场的变化规律}

%%%%%%%%%%%%%%%%%%%%%%%%%%%%%%%%%%%%%%%%%%%%%%%%%%%%%%%%%%%%%%%%
\section{实~~验~~原~~理}
\subsection{霍尔效应}
在一块长方形薄金属板两边对称点1和2之间接一个灵敏电流计(如图1),沿着x轴
正向通以电流I。若在z方向加上磁场B,电流计指针立即偏转,说明1,2两点间产生
了电位差。霍尔发现此电位差与电流强度I和磁感应强度B均成正比,与板厚度成反比,即
\[U_{H}=R_{H}\frac{IB}{d}=K_{H}IB\]
其中$U_{h}$为霍尔电压,$R_{h}$为霍尔系数,$K_{h}=R_{h}/d$为霍尔片的灵敏度.
用洛伦兹力可说明此公式,并可进一步得得到
\[R_{h}=\frac{1}{ne}\]
\[K_{H}=\frac{R_{H}}{d}\]
其中e为载流子电荷,n为载流子浓度.$R_{H}$的单位是$m^{3}/c$.
式(2)和式(3)对大多数金属均成立,但对霍尔系数较高的半导体材料,公式中应引入以一霍尔因子A.在罗磁场的条件下,A=3π/8,故
\[R_{H}=\frac{3\pi}{8}\frac{1}{ne}\]
本次实验为简化计算,A近似为1.
\subsection{霍尔效应的副效应}
实际情况下,会有他一些副效应与霍尔效应混在一起,使霍尔电压的测量产生误差,因此必须尽量消除之.各种效应的特点如下:
\subsubsection{厄廷好森效应}
厄廷好森效应所引起的电位差$U_{E}$是指由于载流子以不同的速度在平行于x轴方向上运动,因此在磁场作用下,速度不同于平均速度的载流子在洛伦兹力与霍尔电场力的共同作用下,向y轴方向的两侧偏移,其动能在霍尔片的两侧转化为热能,在两点间产生温差,从而出现温差电动势$U_{E}$.$U_{E}$正比于IB,正负与I,B的方向有关.
\subsubsection{能撕脱效应}
能撕脱效应所引起的电位差$U_{N}$是指由于连接点3,4处接触电阻不同而产生不同的焦耳热,使3,4两点温度不同,从而使载流子在x方向的运动产生热流,它在磁场作用下在1,2两点间产生电位差$U_{N}$.$U_{N}$的符号与磁场B的方向有关.
\subsubsection{里纪-勒杜克效应}
里纪-勒杜克效应引起的电位差$U_{R}$是指由于上述热流中载流子速度各不相同,在磁场作用下会使1,2两点出现温差电动势$U_{R}$.$U_{R}$方向与B的方向有关.
\subsubsection{不等位效应}
不等位效应引起的电位差$U_{O}$是指由于制作上的困难,1,2两点不可能恰好处在同一条等位线上,因此只要样品中有电流通过,即使B不存在,1,2两点间也会出现电位差$U_{O}$.$U_{O}$的正负只与电流方向有关,其大小在磁场不同时也略不同.
\subsubsection{实际测量}
实际测量时,由于仪表调整的状态,及仪器电压受杂散电磁场和电源地线的影响,电压表会有附加电压$U_{S}$.$U_{S}$与电流,磁场方向无关.


当I,B确定后,霍尔片上的输出电压应为上述几项的代数和
\[U=f(U_{H},U_{E},U_{N},U_{R},U_{O},U_{S})\]
\subsubsection{副效应的消除方法}
通过改变工作电流I的方向和外向磁场B的方向的不同组合测量课消除或减小$U_{N}$,$U_{R}$,$U_{O}$的影响.$U_{E}$的变化与$U_{H}$变化相同,不能用此法22消除,当$U_{E}$等的值都小于$U_{H}$,实验测量中课略去.消除上述副效应的重点是消除不等位效应$U_{O}$.





\subsection{报告中英文缩略语的规范}
文中的英文缩略语应在首次出现时给出中文含义以及英文全称后再使用。例如,全球定位系统(Global Positioning System,GPS)。


\subsection{外文字母}
\subsubsection{斜体外文字母用于表示量的符号,主要用于下列场合}

\begin{enumerate}
\renewcommand{\labelenumi}{(\theenumi)}
\item 变量符号、变动附标及函数。
\item 用字母表示的数及代表点、线、面、体和图形的字母。
\item 特征数符号,如Re (雷诺数)、Fo (傅里叶数)、Al (阿尔芬数)等。
\item 在特定场合中视为常数的参数。
\end{enumerate} 

\subsubsection{正体外文字母用于表示名称及与其有关的代号,主要用于下列场合}
\begin{enumerate}
\renewcommand{\labelenumi}{(\theenumi)}
\item 有定义的已知函数(例如$\sin$, $\exp$, $\ln$等)。
\item 其值不变的数学常数(例如$\mathrm{e} = 2.718 281 8\cdots)$及已定义的算子。
\item 法定计量单位、词头和量纲符号。
\item 数学符号。
\item 化学元素符号。
\item 机具、仪器、设备和产品等的型号、代号及材料牌号。
\item 硬度符号。
\item 不表示量的外文缩写字。
\item 表示序号的拉丁字母。
\item 量符号中为区别其他量而加的具有特定含义的非量符号下角标。
\end{enumerate} 
%%%%%%%%%%%%%%%%%%%%%%%%%%%%%%%%%%%%%%%%%%%%%%%%%%%%%%%%%%%%%%%%


\section{思~~考~~题}

\subsection{如何计算实验中霍尔片载流子迁移率}
由"实验数据处理",$\mu=\Omega R_{H}$,其中$\Omega$为霍尔片的电导率,$R_{H}$为霍尔系数.具体计算过程将数据处理部分.
\subsection{如何观察不等位效应?如何消除不等位效应对测量带来的影响?}
将电磁铁的激励电流切断,B=0,此时霍尔片两侧面间仍有电位差.
改变电流$I_{H}$的方向,此电位差正负改变,这就是观察不等位效应的方法.
测量出$(+B,+I_{H}),(+B,-I_{H}),(-B,I_{H}),(-B,+I_{H})$下的电势差$U_{1},U_{2},U_{3},U_{4}$,并按式$U_{H}=\frac{1}{4}U_{1}-U_{2}+U_{3}-U_{4}$计算即可消除不等位效应,理由在数据处理部分已经给出.
\subsection{如何利用霍尔效应测量磁场?}
给霍尔片通以电流I,测出片上的霍尔电压$U_{H}$,即可通过式$ B=\frac{U_{H}}{K_{H}I} $计算出磁场,$K_{H}$课用实验的方法确定.
%%%%%%%%%%%%%%%%%%%%%%%%%%%%%%%%%%%%%%%%%%%%%%%%%%%%%%%%%%%%%%%%
\section{实~~验~~总~~结}
通过这次实验,我对于霍尔效应有了更加直观,深入的认识.学会了消除霍尔副效应的方法.本次实验内容丰富,各个小实验之间联系密切,前面实验的结构直接影响到了后面的实验.这让我对物理实验的方法有了更深刻的认识,对物理现象内在的联系有了进一步的体会.
%%%%%%%%%%%%%%%%%%%%%%%%%%%%%%%%%%%%%%%%%%%%%%%%%%%%%%%%%%%%%%%%
%  参考文献
%%%%%%%%%%%%%%%%%%%%%%%%%%%%%%%%%%%%%%%%%%%%%%%%%%%%%%%%%%%%%%%%
%  参考文献按GB/T 7714-2015《文后参考文献著录规则》的要求著录. 
%  参考文献在正文中的引用方法:\cite{bib文件条目的第一行}



\end{document}
